\chapter{Automates finis}

\minitoc

\begin{exo}[Exercice 1, exemple]
On considère l'automate \(A_0\) ci-dessous :

\begin{center}
\begin{tikzpicture}
\node[state, initial above] (q1) {\(q_1\)};
\node[state, left of=q1] (q0) {\(q_0\)};
\node[state, below of=q0] (q3) {\(q_3\)};
\node[state, accepting, right of=q1] (q2) {\(q_2\)};
\node[state, accepting, below of=q2] (q4) {\(q_4\)};

\path[->] (q1) edge[above] node {\(a\)} (q0)
          (q1) edge[above] node {\(c\)} (q2)
          (q3) edge[left] node {\(b\)} (q0)
          (q3) edge[loop left] node {\(a\)} (q3)
          (q3) edge[bend right, below] node {\(c\)} (q1)
          (q2) edge[loop above] node {\(b\)} (q2)
          (q2) edge[bend left, right] node {\(a\)} (q4)
          (q2) edge[bend right, left] node {\(c\)} (q4);
\end{tikzpicture}
\end{center}

\begin{enumerate}
    \item Utiliser le vocabulaire approprié pour décrire l'automate \(A_0\). En particulier, donner sa définition formelle et représenter sa table de transition. \\
    \item Donner quelques exemples de mots acceptés par l'automate et de mots rejetés par l'automate. \\
    \item Donner son équivalent complet puis l'émonder.
\end{enumerate}
\end{exo}

\begin{corr}
\note{À venir}
\end{corr}

\begin{exo}[Exercice 2, déterminisation]
On considère l'automate \(A_1\) ci-dessous :

\begin{center}
\begin{tikzpicture}
\node[state, initial] (q0) {\(q_0\)};
\node[state, below right of=q0] (q2) {\(q_2\)};
\node[state, above right of=q2] (q1) {\(q_1\)};
\node[state, right of=q1, accepting] (q3) {\(q_3\)};

\path[->] (q0) edge[loop above] node {\(a\)} (q0)
          (q0) edge[above] node {\(a\)} (q1)
          (q0) edge[below] node {\(b\)} (q2)
          (q2) edge[loop below] node {\(a,b\)} (q2)
          (q2) edge[above] node {\(a\)} (q1)
          (q1) edge[above] node {\(b\)} (q3);
\end{tikzpicture}
\end{center}

\begin{enumerate}
    \item Justifier que l'automate \(A_1\) n'est pas déterministe. \\
    \item Donner un mot \(m\) pouvant donner lieu à au moins deux calculs, l'un aboutissant à un état final, l'autre aboutissant à un état non-final. Ce mot est-il reconnu par l'automate ? \\
    \item Déterminer cet automate pour obtenir \(A_{1\,\det}\). Le mot \(m\) est-il reconnu par \(A_{1\,\det}\) ?
\end{enumerate}
\end{exo}

\begin{corr}
\note{À venir}
\end{corr}

\begin{exo}[Exercice 3, automates \guillemets{évidents}]
On considère l'alphabet \(\Sigma=\accol{a,b,c}\).

Pour chaque ensemble de mots \(L_i\), donner une expression régulière \(r_i\) dénotant \(L_i\) et dessiner un automate \(A_i\) (déterministe ou non) reconnaissant \(L_i\) :

\begin{enumerate}
    \item \(L_1\) : les mots de trois lettres qui commencent par \(a\) ; \\
    \item \(L_2\) : les mots qui ne contiennent qu'un seul \(b\) ; \\
    \item \(L_3\) : les mots qui ont \(ab\) pour préfixe ; \\
    \item \(L_4\) : les mots qui ont \(bac\) pour suffixe ; \\
    \item \(L_5\) : les mots qui ont \(abba\) pour facteur ; \\
    \item \(L_6\) : les mots qui ont \(abba\) pour sous-mot.
\end{enumerate}
\end{exo}

\begin{corr}
\note{À venir}
\end{corr}

\begin{exo}[Exercice 4, donner un sens aux états]
On considère l'alphabet \(\Sigma=\accol{0,1}\). Représenter un automate reconnaissant :

\begin{enumerate}
    \item l'ensemble des mots contenant un nombre pair de symboles ; \\
    \item l'ensemble des mots tels que le nombre d'occurrences de \(1\) soit multiple de \(3\) ; \\
    \item l'ensemble des mots \(m\) tels que \(\abs{m}_0-\abs{m}_1\equiv1\croch{9}\).
\end{enumerate}
\end{exo}

\begin{corr}
\note{À venir}
\end{corr}

\begin{exo}[Exercice 5, automate déterministe complet]
On définit un automate déterministe incomplet comme un quintuplet \[A=\paren{Q,\Sigma,q_i,Q_F,\delta:D\subset Q\times\Sigma\to Q}.\]

\begin{enumerate}
    \item Définir formellement l'automate complet \(A_c\) associé. \\
    \item Justifier formellement que ces deux automates sont équivalents.
\end{enumerate}
\end{exo}

\begin{corr}
\note{À venir}
\end{corr}

\begin{exo}[Exercice 6, automate produit]
On considère l'alphabet \(\Sigma=\accol{a,b}\).

\begin{enumerate}
    \item Représenter un automate déterministe \(A\) reconnaissant les mots ayant un nombre pair de symboles. \\
    \item Représenter un automate déterministe \(A\prim\) reconnaissant les mots composés de symboles \(a\) puis de symboles \(b\). \\
    \item En utilisant l'automate produit, représenter un automate reconnaissant les mots contenant un nombre pair de symboles qui sont composés de symboles \(a\) puis de symboles \(b\). Donner un sens à chaque état de l'automate obtenu.
\end{enumerate}
\end{exo}

\begin{corr}
\note{À venir}
\end{corr}

\begin{exo}[Exercice 7, déterminisation \guillemets{explosive}]
On considère l'alphabet \(\Sigma=\accol{a,b}\) et \(n\in\Ns\). On pose \(L_n\) le langage des mots ayant un \(a\) \guillemets{\(n\) lettres avant la fin du mot}. Par exemple, \(L_1\) est le langage des mots ayant un \(a\) une lettre avant la fin, \ie en dernière lettre ; \(L_2\) est le langage des mots ayant un \(a\) deux lettres avant la fin, \ie en avant-dernière lettre.

\begin{enumerate}[series=ex2.7]
    \item Donner une expression régulière \(e_n\) dénotant \(L_n\). \\
    \item Donner un automate \(A_n\) non-déterministe à \(n+1\) états qui reconnaît \(L_n\). On nommera \(q_0,\dots,q_n\) les états. \\
    \item Déterminiser \(A_3\). Combien d'états l'automate obtenu a-t-il ? \\
    \item Trouver un mot de trois lettres permettant, depuis \(q_0\), d'atteindre exactement l'ensemble des états \(\accol{q_0,q_2}\).
\end{enumerate}

Plus généralement, on va montrer que l'automate déterminisé associé à \(A_n\) possède \(2^n\) états.

Pour \(m\in\Sigma\etoile\) et \(q\) un état, on note \(A\paren{m}\) l'ensemble des états de \(A_n\) accessibles à la lecture du mot \(m\) et \(L_{\to q}\) le langage reconnu par l'automate \(A_n\) légèrement modifié de sorte que \(q\) soit son unique état final.

\begin{enumerate}[resume=ex2.7]
    \item Décrire en français les langages \(L_{\to q_0},L_{\to q_1},\dots,L_{\to q_n}\). \\
    \item Montrer que pour \(m\in\Sigma\etoile\), \(A\paren{m}=\accol{q\tq m\in L_{\to q}}\). \\
    \item Montrer que pour chaque ensemble d'états \(Q\) contenant \(q_0\), il existe un mot \(m\) tel que \(\abs{m}=n\) et \(A\paren{m}=Q\). \\
    \item En déduire, le nombre d'états de l'automate déterminisé associé à \(A_n\).
\end{enumerate}

Peut-être que la méthode de déterminisation construit un automate inutilement compliqué... Est-il possible de trouver un automate déterministe de moins de \(2^n\) états qui reconnaisse \(L_n\) ? Ce qui suit montre que non.

Supposons qu'il existe un automate déterministe complet ayant strictement moins de \(2^n\) états qui reconnaît \(L_n\).

\begin{enumerate}[resume=ex2.7]
    \item Justifier qu'il existe deux mots \(u\) et \(v\) de longueur \(n\) dont la lecture aboutit au même état. \\
    \item En considérant la première lettre qui différencie ces deux mots, et en les rallongeant, aboutir à une contradiction.
\end{enumerate}
\end{exo}

\begin{corr}
\note{À venir}
\end{corr}

\begin{exo}[Exercice 8, lemme d'Arden et application]
Lemme d'Arden : si \(L\) est un langage vérifiant l'égalité \(L=E.L\union F\) où \(E,F\) sont des langages tels que \(\epsilon\not\in E\), alors \(L=E\etoile.F\).

On peut rencontrer la formulation suivante : \(E\etoile.F\) est l'unique solution de l'équation \(X=E.X\union F\) d'inconnue \(X\), lorsque \(\epsilon\not\in E\).

\begin{enumerate}[series=ex2.8]
    \item Démontrer le lemme d'Arden (version équation). Montrer que le langage proposé est bien solution de l'équation puis montrer que c'est l'unique solution de l'équation.
\end{enumerate}

Application : le lemme d'Arden peut servir à exprimer le langage reconnu par un automate sous forme d'expression régulière.

Exemple : on considère l'automate \(A\) ci-dessous :

\begin{center}
\begin{tikzpicture}
\node[state, initial] (q1) {\(q_1\)};
\node[state, below right of=q1, accepting] (q3) {\(q_3\)};
\node[state, above right of=q3] (q2) {\(q_2\)};

\path[->] (q1) edge[above] node {\(a\)} (q2)
          (q1) edge[below] node {\(b\)} (q3)
          (q3) edge[loop below] node {\(a,b\)} (q3)
          (q2) edge[loop right] node {\(b\)} (q2)
          (q2) edge[below] node {\(a\)} (q3);
\end{tikzpicture}
\end{center}

Pour \(i\in\accol{1,2,3}\), on note \(L_i\) le langage reconnu en prenant \(q_i\) pour état initial.

\begin{enumerate}[resume=ex2.8]
    \item Pour chaque état \(q_i\), établir une relation entre les \(L_i\). \\
    \item À l'aide du lemme d'Arden, donner une expression pour les langages \(L_i\). \\
    \item En déduire le langage reconnu par l'automate.
\end{enumerate}
\end{exo}

\begin{corr}
\note{À venir}
\end{corr}

\begin{exo}[Exercice 9, langages non-rationnels]
Montrer que les langages ci-dessous ne sont pas rationnels :

\begin{enumerate}
    \item \(L_1=\accol{a^nb^n\tq n\in\N}\) ; \\
    \item \(L_2=\accol{a^p\tq p\text{ est premier}}\) ; \\
    \item \(L_3=\accol{m\tq\abs{m}_a=\abs{m}_b}\).
\end{enumerate}
\end{exo}

\begin{corr}
\note{À venir}
\end{corr}

\begin{exo}[Exercice 10, vers l'automate de Glushkov : automate local]
Montrer que la procédure de construction de l'automate local \(A_\loc\) associé à un langage local \(LL\) est correcte, \cad que \(\Lendo{A_\loc}=LL\).
\end{exo}

\begin{corr}
\note{À venir}
\end{corr}

\begin{exo}[Exercice 11, vers l'automate de Glushkov : ensembles \guillemets{caractéristiques}]
Pour les langages suivants, déterminer les ensembles caractéristiques \(P\paren{L}\), \(S\paren{L}\), \(F\paren{L}\) et \(N\paren{L}\), puis déterminer si le langage est local :

\begin{enumerate}
    \item \(L_1=\Lendo{abab}\) ; \\
    \item \(L_2=\Lendo{abc}\) ; \\
    \item \(L_3=\Lendo{a\etoile}\) ; \\
    \item \(L_4=\Lendo{\paren{ab}\etoile}\) ; \\
    \item \(L_5=\Lendo{\paren{ab}\etoile a\etoile}\).
\end{enumerate}
\end{exo}

\begin{corr}
\note{À venir}
\end{corr}

\begin{exo}[Exercice 12, vers l'automate de Glushkov : propriétés des langages linéaires]
\begin{enumerate}
    \item Soit \(L\) un langage local. Exprimer \(P\paren{L\etoile}\), \(S\paren{L\etoile}\) et \(F\paren{L\etoile}\) en fonction de \(P\paren{L}\), \(S\paren{L}\) et \(F\paren{L}\) ; montrer que \(L\etoile\) est un langage local. \\
    \item De même, montrer que l'union de deux langages locaux sur des alphabets distincts est un langage local. \\
    \item De même, montrer que la concaténation de deux langages locaux sur des alphabets distincts est un langage local. \\
    \item Montrer que \(\ensvide\), \(\accol{\epsilon}\) et \(\accol{a}\) avec \(a\in\Sigma\) sont des langages locaux. \\
    \item En déduire que le langage dénoté par une expression régulière linéaire est local.
\end{enumerate}
\end{exo}

\begin{corr}
\note{À venir}
\end{corr}
