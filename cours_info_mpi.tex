% Set up the document's format to A4 and the font's size to 12pt.
\documentclass[a4paper,12pt]{report}

% Set up the document's title, author and date.
\title{Informatique -- MPI}
\author{Romain Bricout}
\date{\today}

% Set up the input's encoding to UTF-8, the document's font and language to T1 (adapted to french) and french (the grammar linter uses this parameter).
\usepackage[utf8]{inputenc}
\usepackage[T1]{fontenc}
\usepackage[frenchb]{babel}

\usepackage[dvipsnames]{xcolor}

% Set up the document's margins.
\usepackage{geometry}
\geometry{hmargin=1.5cm,vmargin=1.5cm}

% The three main maths packages. They are used for a lot of things.
\usepackage{amssymb,amsmath}
\usepackage{mathtools}

% Useful to create nice and easy signs or variations tables.
\usepackage{tkz-tab}

% Useful to create any kind of visual representation (graph functions, illustrate geometry problems, etc)
\usepackage{tikz}
\usetikzlibrary{patterns,angles,quotes,arrows,arrows.meta,bending,matrix,calc,positioning,automata}
\tikzset{->,>=stealth',node distance=3cm,every state/.style={thick,fill=gray!10},initial text=$ $}

% Allows to edit the itemize environment's default item document-wide.
\usepackage{enumitem}

% Allows to define \notfoo or \nfoo (not recommended) in order for \not\foo to work as wished.
\usepackage{newtxmath}
\DeclareSymbolFont{CMletters}{OML}{cmm}{m}{it}
\DeclareMathSymbol{\nu}{\mathord}{CMletters}{23}

% Makes the table of contents clickable and gives useful commands for links in general.
\usepackage[hypertexnames=false]{hyperref}
\hypersetup{colorlinks=false,linktoc=all}

% Gives the llbracket and rrbracket commands for integer intervals.
\usepackage{stmaryrd}

% Useful to insert nice-looking quotes.
\usepackage{epigraph}

% Allows to insert chapter-specific table of contents.
\usepackage{minitoc}
\mtcselectlanguage{french}
\setcounter{minitocdepth}{6}

% Useful when units are needed.
\usepackage{siunitx}
\sisetup{
locale=FR,
detect-all,
inter-unit-product=\ensuremath{\cdot},
list-final-separator={et},
list-pair-separator={et},
range-phrase={\ensuremath{\xleftrightarrow{}}},
exponent-product=\ensuremath{\cdot},
per-mode=power-positive-first
}

\usepackage[thmmarks,hyperref]{ntheorem}
\makeatletter
\let\old@thm\@thm
\usepackage[lowercase]{theoremref}
\def\@thm#1#2#3{\def\thmref@currname{#3}\old@thm{#1}{#2}{#3}}
\makeatother

% Allows whiteboard digits with \mathds
\usepackage{dsfont}

\usepackage{needspace}

% Useful for better-looking oneline fractions
\usepackage{nicefrac}

% Set up the horizontal space before the first line of a new paragraph to 2em and the vertical space between two paragraphs to 1em.
\setlength{\parindent}{0pt}
\setlength{\parskip}{1em}

% Adds 0.5em to the vertical space between two lines in an align environment. It looks better.
\addtolength{\jot}{0.5em}

% Allows align environment to break if it's too long to fit in the page where it began.
\allowdisplaybreaks[1]

% Trick to make semicolons considered like relation operators (such as =) and therefore being equidistantly spaced from the two numbers around it.
\mathcode`;=\numexpr\mathcode`;-"3000

% Commands for size-adaptative parentheses, brackets, curly brackets, absolute value and magnitude.
\newcommand{\paren}[1]{\left(#1\right)} % (x)
\newcommand{\croch}[1]{\left[#1\right]} % [x]
\newcommand{\accol}[1]{\left\lbrace#1\right\rbrace} % {x}
\newcommand{\abs}[1]{\left\lvert#1\right\rvert} % |x|
\newcommand{\floor}[1]{\left\lfloor#1\right\rfloor} % ⌊x⌋
\newcommand{\ceil}[1]{\left\lceil#1\right\rceil} % ⌈x⌉

% Commands for size-adaptative intervals and integer intervals. The commands' roots are "interv" and "interventier" and the added e or i at the end mean "excluded" and "included" respectively.
\newcommand{\intervii}[2]{\left[#1;#2\right]} % [a;b]
\newcommand{\intervee}[2]{\left]#1;#2\right[} % ]a;b[
\newcommand{\intervie}[2]{\left[#1;#2\right[} % [a;b[
\newcommand{\intervei}[2]{\left]#1;#2\right]} % ]a;b]
\newcommand{\interventierii}[2]{\left\llbracket#1;#2\right\rrbracket} % non-ASCII characters needed
\newcommand{\interventieree}[2]{\left\rrbracket#1;#2\right\llbracket} % non-ASCII characters needed
\newcommand{\interventierie}[2]{\left\llbracket#1;#2\right\llbracket} % non-ASCII characters needed
\newcommand{\interventierei}[2]{\left\rrbracket#1;#2\right\rrbracket} % non-ASCII characters needed

% Commands for usually used sets.
\newcommand{\N}{\mathbb{N}} % natural integers
\newcommand{\Ns}{\mathbb{N}^*}

\newcommand{\Z}{\mathbb{Z}} % relative integers
\newcommand{\Zp}{\mathbb{Z}_+}
\newcommand{\Zs}{\mathbb{Z}^*}
\newcommand{\Zps}{\mathbb{Z}_+^*}

\newcommand{\D}{\mathbb{D}} % decimal numbers
\newcommand{\Dp}{\mathbb{D}_+}
\newcommand{\Dm}{\mathbb{D}_-}
\newcommand{\Ds}{\mathbb{D}^*}
\newcommand{\Dps}{\mathbb{D}_+^*}
\newcommand{\Dms}{\mathbb{D}_-^*}

\newcommand{\Q}{\mathbb{Q}} % rational numbers
\newcommand{\Qp}{\mathbb{Q}_+}
\newcommand{\Qm}{\mathbb{Q}_-}
\newcommand{\Qs}{\mathbb{Q}^*}
\newcommand{\Qps}{\mathbb{Q}_+^*}
\newcommand{\Qms}{\mathbb{Q}_-^*}

\newcommand{\R}{\mathbb{R}} % real numbers
\newcommand{\Rp}{\mathbb{R}_+}
\newcommand{\Rm}{\mathbb{R}_-}
\newcommand{\Rs}{\mathbb{R}^*}
\newcommand{\Rps}{\mathbb{R}_+^*}
\newcommand{\Rms}{\mathbb{R}_-^*}
\newcommand{\Rb}{\overline{\mathbb{R}}}

\newcommand{\C}{\mathbb{C}} % complex numbers
\newcommand{\Cs}{\mathbb{C}^*}

\newcommand{\K}{\mathbb{K}}
\newcommand{\Ks}{\mathbb{K}^*}

\newcommand{\A}{\mathbb{A}}
\renewcommand{\L}[2]{\mathscr{L}\paren{#1,#2}}
\newcommand{\Lc}[2]{\mathscr{L}_c\paren{#1,#2}}
\newcommand{\Lendo}[1]{\mathscr{L}\paren{#1}}

\newcommand{\prem}{\mathbb{P}}

\newcommand{\U}{\mathbb{U}} % complex numbers whose modulus is 1

\renewcommand{\P}[1]{\mathscr{P}\paren{#1}} % subsets of a set
\newcommand{\Pf}[1]{\mathscr{P}_f\paren{#1}} % finite subsets of a set
\newcommand{\F}[2]{\mathscr{F}\paren{#1,#2}} % functions from 1 to 2
\newcommand{\V}[1]{\mathscr{V}\paren{#1}} % neighborhood of a number

% Redefines \Re and \Im to print Re and Im (the same way as ln or lim) instead of fraktur R and I which don't look nice and are less readable.
\renewcommand{\Re}{\operatorname{Re}}
\renewcommand{\Im}{\operatorname{Im}}
\newcommand{\Card}{\operatorname{Card}}

% Command to print an upright e for the exponential instead of a slanted e and put the exponent.
\newcommand{\e}[1]{\mathrm{e}^{#1}}

% Command to print the imaginary i with a little space on the right. This way, the exponents don't look confusing. \i normally prints a dotless i.
\renewcommand{\i}{i\mkern1mu}

% Redefines \vec such that the arrow covers the whole name of the vector.
%\renewcommand{\vec}[1]{\overrightarrow{#1}}

% Commands for 2D and 3D vectors' coordinates
\newcommand{\dcoords}[2]{\begin{pmatrix}#1\\#2\end{pmatrix}}
\newcommand{\tcoords}[3]{\begin{pmatrix}#1\\#2\\#3\end{pmatrix}}

% Redefines binom to print nicer parentheses around the numbers.
\renewcommand{\binom}[2]{\begin{pmatrix}#2\\#1\end{pmatrix}}

% Command for a QED black square. It automatically prints a whitespace before the square such that it looks nice.
\newcommand{\cqfd}{\text{ }\blacksquare}

% Commands with more explicit names for the best way to express divisibility (mid and nmid).
\newcommand{\divise}{\mid}
\newcommand{\notdivise}{\nmid}

% Commands that do the exact same thing but with explicit names for a complex number's conjugate and an event's negation in probability.
\newcommand{\conj}[1]{\overline{#1}}

% Command for a size-adaptative middle bar meaning "such that" (with spacing around it in order to look nice).
\newcommand{\tq}{\;\middle|\;}

% Command with an explicit name for the scalar product.
\newcommand{\scal}{\cdot}
\newcommand{\vecto}{\operatorname{_\wedge}}

% Shortcut for forcing displaystyle in inline mode.
\newcommand{\ds}{\displaystyle}

% Make the not version of implies, impliedby and iff look nice.
\newcommand{\notimp}{\centernot{\imp}}
\newcommand{\notimpr}{\centernot{\impr}}
\newcommand{\notssi}{\centernot{\ssi}}

\renewcommand{\subset}{\subseteq}
\renewcommand{\supset}{\supseteq}
\newcommand{\notsubset}{\centernot{\subset}}
\newcommand{\notsupset}{\centernot{\supset}}

% Shortcut for P(event).
\newcommand{\proba}[1]{\mathbb{P}\paren{#1}}
\newcommand{\probacond}[2]{\mathbb{P}_{#2}\paren{#1}}

% More explicit names for land (logical and) and lor (logical or).
\newcommand{\et}{\land}
\newcommand{\ou}{\lor}
\newcommand{\non}{\lnot}

% Explicitly named environment for tkz-tab tables. Automatically centers the table and handles the tikzpicture environment.
\newenvironment{tkz}
{
\begin{tikzpicture}
}
{
\end{tikzpicture}
}

% More explicitly named commands for the creation of tkz-tab tables.
\newcommand{\tableauinit}[2]{\tkzTabInit{#1}{#2}}
\newcommand{\tableausignes}[1]{\tkzTabLine{#1}}
\newcommand{\tableauvariations}[1]{\tkzTabVar{#1}}

% Shortcut for the curve and the domain of the given function.
\newcommand{\graphe}[1]{\Gamma_{#1}}
\newcommand{\ensembledef}[1]{\mathcal{D}_{#1}}

\renewcommand{\S}[1]{\mathfrak{S}_{#1}}
\newcommand{\frakA}[1]{\mathfrak{A}_{#1}}

\newcommand{\semihrule}{\rule{256.074815pt}{0.4pt}}

% Various environments that create boxes. Each one is one type of thing (example, proof, etc). Each type has its own automatic counter.
\theoremstyle{break}
\theorembodyfont{\upshape}
\theoremheaderfont{\itshape}
\theoremprework{\bigskip\needspace{\baselineskip}\color{green}\hrule\color{black}}
\theorempostwork{\bigskip}
\newtheorem{rem}{Remarque}[chapter]

\theoremstyle{break}
\theorembodyfont{\upshape}
\theoremheaderfont{\itshape}
\theoremprework{\bigskip\needspace{\baselineskip}\color{green}\hrule\color{black}}
\theorempostwork{\bigskip}
\newtheorem{ex}[rem]{Exemple}

\theoremstyle{break}
\theorembodyfont{\upshape}
\theoremheaderfont{\itshape}
\theoremprework{\bigskip\needspace{\baselineskip}\color{green}\hrule\color{black}}
\theorempostwork{\bigskip}
\newtheorem{rappel}[rem]{Rappel}

\theoremstyle{break}
\theorembodyfont{\upshape}
\theoremheaderfont{\itshape}
\theoremprework{\bigskip\needspace{\baselineskip}\color{brown}\hrule\color{black}}
\theorempostwork{\bigskip}
\newtheorem{oubli}[rem]{Oubli}

\theoremstyle{break}
\theorembodyfont{\upshape}
\theoremheaderfont{\normalfont\bfseries}
\theoremprework{\bigskip\needspace{\baselineskip}\color{blue}\hrule\color{black}}
\theorempostwork{\bigskip}
\newtheorem{defi}[rem]{Définition}

\theoremstyle{break}
\theorembodyfont{\upshape}
\theoremheaderfont{\normalfont\bfseries}
\theoremprework{\bigskip\needspace{\baselineskip}\color{blue}\hrule\color{black}}
\theorempostwork{\bigskip}
\newtheorem{reform}[rem]{Reformulation}

\theoremstyle{break}
\theorembodyfont{\upshape}
\theoremheaderfont{\normalfont\bfseries}
\theoremprework{\bigskip\needspace{\baselineskip}\color{magenta}\hrule\color{black}}
\theorempostwork{\bigskip}
\newtheorem{exo}[rem]{Exercice}

\theoremstyle{break}
\theorembodyfont{\upshape}
\theoremheaderfont{\normalfont\bfseries}
\theoremprework{\bigskip\needspace{\baselineskip}\color{magenta}\hrule\color{black}}
\theorempostwork{\bigskip}
\newtheorem{exos}[rem]{\(\star\) Exercice}

\theoremstyle{break}
\theorembodyfont{\upshape}
\theoremheaderfont{\normalfont\bfseries}
\theoremprework{\bigskip\needspace{\baselineskip}\color{magenta}\hrule\color{black}}
\theorempostwork{\bigskip}
\newtheorem{exoss}[rem]{\(\star\star\) Exercice}

\theoremstyle{break}
\theorembodyfont{\upshape}
\theoremheaderfont{\normalfont\bfseries}
\theoremprework{\bigskip\needspace{\baselineskip}\color{magenta}\hrule\color{black}}
\theorempostwork{\bigskip}
\newtheorem{exosss}[rem]{\(\star\star\star\) Exercice}

\theoremstyle{break}
\theorembodyfont{\upshape}
\theoremheaderfont{\normalfont\bfseries}
\theoremprework{\bigskip\needspace{\baselineskip}\color{magenta}\hrule\color{black}}
\theorempostwork{\bigskip}
\newtheorem{exops}[rem]{\(+\star\) Exercice}

\theoremstyle{break}
\theorembodyfont{\upshape}
\theoremheaderfont{\normalfont\bfseries}
\theoremprework{\bigskip\needspace{\baselineskip}\color{magenta}\hrule\color{black}}
\theorempostwork{\bigskip}
\newtheorem{exopss}[rem]{\(+\star\star\) Exercice}

\theoremstyle{break}
\theorembodyfont{\upshape}
\theoremheaderfont{\normalfont\bfseries}
\theoremprework{\bigskip\needspace{\baselineskip}\color{magenta}\hrule\color{black}}
\theorempostwork{\bigskip}
\newtheorem{exopsss}[rem]{\(+\star\star\star\) Exercice}

\theoremstyle{break}
\theorembodyfont{\upshape}
\theoremheaderfont{\normalfont\bfseries}
\theoremprework{\bigskip\needspace{\baselineskip}\color{magenta}\semihrule\color{green}\semihrule\color{black}}
\theorempostwork{\bigskip}
\newtheorem{exoex}[rem]{Exercice/Exemple}

\theoremstyle{break}
\theorembodyfont{\upshape}
\theoremheaderfont{\normalfont\bfseries}
\theoremprework{\bigskip\needspace{\baselineskip}\color{blue}\semihrule\color{red}\semihrule\color{black}}
\theorempostwork{\bigskip}
\newtheorem{defprop}[rem]{Définition/Proposition}

\theoremstyle{break}
\theorembodyfont{\upshape}
\theoremheaderfont{\normalfont\bfseries}
\theoremprework{\bigskip\needspace{\baselineskip}\color{blue}\semihrule\color{red}\semihrule\color{black}}
\theorempostwork{\bigskip}
\newtheorem{deftheo}[rem]{Définition/Théorème}

\theoremstyle{break}
\theorembodyfont{\upshape}
\theoremheaderfont{\normalfont\bfseries}
\theoremprework{\bigskip\needspace{\baselineskip}\color{blue}\hrule\color{black}}
\theorempostwork{\bigskip}
\newtheorem{nota}[rem]{Notation}

\theoremstyle{break}
\theorembodyfont{\upshape}
\theoremheaderfont{\itshape}
\theoremprework{\bigskip\needspace{\baselineskip}\color{blue}\hrule}
\theorempostwork{\hrule\color{black}\needspace{\baselineskip}\bigskip}
\newtheorem*{brouill}{Brouillon}

\theoremstyle{break}
\theorembodyfont{\itshape}
\theoremheaderfont{\normalfont\bfseries}
\theoremprework{\bigskip\needspace{\baselineskip}\color{red}\hrule\color{black}}
\theorempostwork{\bigskip}
\newtheorem{theo}[rem]{Théorème}

\theoremstyle{break}
\theorembodyfont{\itshape}
\theoremheaderfont{\normalfont\bfseries}
\theoremprework{\bigskip\needspace{\baselineskip}\color{red}\hrule\color{black}}
\theorempostwork{\bigskip}
\newtheorem{prop}[rem]{Proposition}

\theoremstyle{break}
\theorembodyfont{\itshape}
\theoremheaderfont{\normalfont\bfseries}
\theoremprework{\bigskip\needspace{\baselineskip}\color{red}\hrule\color{black}}
\theorempostwork{\bigskip}
\newtheorem{cor}[rem]{Corollaire}

\theoremstyle{break}
\theorembodyfont{\itshape}
\theoremheaderfont{\normalfont\bfseries}
\theoremprework{\bigskip\needspace{\baselineskip}\color{red}\hrule\color{black}}
\theorempostwork{\bigskip}
\newtheorem{lem}[rem]{Lemme}

\theoremstyle{break}
\theorembodyfont{\upshape}
\theoremheaderfont{\normalfont\bfseries}
\theoremprework{\bigskip\needspace{\baselineskip}\color{violet}\hrule\color{black}}
\theorempostwork{\bigskip}
\newtheorem{meth}[rem]{Méthode}

\theoremstyle{break}
\theorembodyfont{\upshape}
\theoremheaderfont{\normalfont\bfseries}
\theoremprework{\bigskip\needspace{\baselineskip}\color{violet}\hrule\color{black}}
\theorempostwork{\bigskip}
\newtheorem{appl}[rem]{Application}

\theoremstyle{break}
\theorembodyfont{\upshape}
\theoremheaderfont{\normalfont\bfseries}
\theoremprework{\bigskip\needspace{\baselineskip}\color{violet}\hrule\color{black}}
\theorempostwork{\bigskip}
\newtheorem{abus}[rem]{Abus}

\theoremstyle{break}
\theorembodyfont{\upshape}
\theoremheaderfont{\normalfont\bfseries}
\theoremprework{\bigskip\needspace{\baselineskip}\color{violet}\hrule\color{black}}
\theorempostwork{\bigskip}
\newtheorem{algo}[rem]{Algorithme}

\theoremstyle{break}
\theorembodyfont{\upshape}
\theoremheaderfont{\normalfont\bfseries}
\theoremprework{\bigskip\needspace{\baselineskip}\color{violet}\hrule\color{black}}
\theorempostwork{\bigskip}
\newtheorem{bilan}[rem]{Bilan}

\theoremstyle{break}
\theorembodyfont{\upshape}
\theoremheaderfont{\itshape}
\theoremprework{\bigskip\needspace{\baselineskip}\color{BurntOrange}\hrule\color{black}}
\theorempostwork{\bigskip}
\newtheorem{corr}[rem]{Correction}

\theoremstyle{break}
\theorembodyfont{\upshape}
\theoremheaderfont{\itshape}
\theoremsymbol{\ensuremath{\cqfd}}
\theoremprework{\bigskip\needspace{\baselineskip}\color{yellow}\hrule\color{black}}
\theorempostwork{\bigskip}
\newtheorem{dem}[rem]{Démonstration}

% Commands to make proofs easier to write
\newcommand{\impdir}{\fbox{\(\imp\)}~}
\newcommand{\imprec}{\fbox{\(\impr\)}~}
\newcommand{\incdir}{\fbox{\(\subset\)}~}
\newcommand{\increc}{\fbox{\(\supset\)}~}
\newcommand{\leqbox}{\fbox{\(\leq\)}~}
\newcommand{\geqbox}{\fbox{\(\geq\)}~}
\newcommand{\unicite}{\fbox{unicité}~}
\newcommand{\existence}{\fbox{existence}~}
\newcommand{\analyse}{\fbox{analyse}~}
\newcommand{\synthese}{\fbox{synthèse}~}
\newcommand{\conclusion}{\fbox{conclusion}~}

\renewcommand{\to}{\longrightarrow}
\renewcommand{\mapsto}{\longmapsto}

\newcommand{\fonction}[5]{\begin{array}[t]{cccc}#1 : & #2 & \to & #3 \\ & #4 & \mapsto & #5\end{array}}
\newcommand{\fonctionlambda}[4]{\begin{array}[t]{ccc}#1 & \to & #2 \\ #3 & \mapsto & #4\end{array}}

\renewcommand{\leq}{\leqslant}
\renewcommand{\geq}{\geqslant}

\newcommand{\pinf}{+\infty}
\newcommand{\minf}{-\infty}

\newcommand{\id}[1]{\mathrm{id}_{#1}}

\renewcommand{\phi}{\varphi}
\renewcommand{\epsilon}{\varepsilon}

\newcommand{\ind}[1]{\mathds{1}_{#1}}

\newcommand{\iR}{\i\R}

\newcommand{\tcheby}[2]{T_{#1}\paren{#2}}
\newcommand{\utcheby}[2]{U_{#1}\paren{#2}}

\mathcode`l="8000
\begingroup
\makeatletter
\lccode`\~=`\l
\DeclareMathSymbol{\lsb@l}{\mathalpha}{letters}{`l}
\lowercase{\gdef~{\ifnum\the\mathgroup=\m@ne \ell \else \lsb@l \fi}}%
\endgroup

\newcommand{\ensvide}{\varnothing}

\newcommand{\rond}{\circ}

\newcommand{\union}{\cup}
\newcommand{\inter}{\cap}
\newcommand{\bigunion}{\bigcup}
\newcommand{\biginter}{\bigcap}

\newcommand{\ssi}{\iff}
\newcommand{\imp}{\implies}
\newcommand{\impr}{\impliedby}

\newcommand{\excluant}{\setminus}

\newcommand{\littletaller}{\mathchoice{\vphantom{\big|}}{}{}{}}
\newcommand{\restr}[2]{{
\left.\kern-\nulldelimiterspace#1\littletaller\right|_{#2}
}}
\newcommand{\corestr}[2]{{
\left.\kern-\nulldelimiterspace#1\littletaller\right|^{#2}
}}
\newcommand{\restrbar}[1]{{
\left.\kern-\nulldelimiterspace#1\littletaller\right|
}}

\newcommand{\rel}{\mathscr{R}}

\newcommand{\classesdequiv}[1]{\nicefrac{#1}{\sim}}

\newcommand{\majo}[1]{\mathrm{majorants}\paren{#1}}
\newcommand{\mino}[1]{\mathrm{minorants}\paren{#1}}

\newcommand{\ensdiv}[1]{\operatorname{div}\paren{#1}}

\newcommand{\E}[1]{\times 10^{#1}}

\setcounter{secnumdepth}{3}

\newcommand{\guillemets}[1]{\og #1 \fg{}}

\newcommand{\prim}{^{\,\prime}}
\newcommand{\seconde}{^{\,\prime\prime}}

\newcommand{\note}[1]{\textbf{\(\star\star\) #1 \(\star\star\)}}
\newcommand{\cad}{c'est-à-dire }
\newcommand{\Cad}{C'est-à-dire }
\newcommand{\ie}{\textit{i.e.} }
\newcommand{\cf}{\textit{cf.} }
\newcommand{\Cf}{\textit{Cf.} }

\usepackage{xparse}

\NewDocumentCommand{\quantifs}{>{\SplitList{;}}m}{\ProcessList{#1}{\insertquantif}}
\newcommand{\insertquantif}[1]{#1,\;\:}

\DeclareDocumentCommand{\groupe}{m O{+}}{\paren{#1,#2}}
\DeclareDocumentCommand{\anneau}{m O{+} O{\times}}{\paren{#1,#2,#3}}
\DeclareDocumentCommand{\corps}{m O{+} O{\times}}{\paren{#1,#2,#3}}

\DeclareDocumentCommand{\poly}{O{\K} O{X}}{#1\croch{#2}}
\DeclareDocumentCommand{\polydeg}{O{\K} m O{X}}{#1_{#2}\croch{#3}}
\DeclareDocumentCommand{\fracrat}{O{\K} O{X}}{#1\paren{#2}}

\DeclareDocumentCommand{\M}{m O{\K}}{\mathscr{M}_{#1}\paren{#2}}
\DeclareDocumentCommand{\sym}{m O{\K}}{\mathscr{S}_{#1}\paren{#2}}
\DeclareDocumentCommand{\antisym}{m O{\K}}{\mathscr{A}_{#1}\paren{#2}}
\DeclareDocumentCommand{\GL}{m O{\K}}{\operatorname{GL}_{#1}\paren{#2}}
\DeclareDocumentCommand{\SL}{m O{\K}}{\operatorname{SL}_{#1}\paren{#2}}
\DeclareDocumentCommand{\Mat}{O{\fami{B}} m}{\operatorname{Mat}_{#1}\paren{#2}}
\newcommand{\pass}[2]{\mathscr{P}_{#1\to#2}}

\DeclareDocumentCommand{\contm}{O{\intervii{a}{b}} O{\K}}{\classe{0}_m\paren{#1,#2}}
\DeclareDocumentCommand{\Esc}{O{\intervii{a}{b}} O{\K}}{\operatorname{Esc}\paren{#1,#2}}

\usepackage{witharrows}

\newcommand{\croix}{^{\times}}

\usepackage{polynom}

\newcommand{\classe}[1]{\mathscr{C}^{#1}}
\newcommand{\ensclasse}[3]{\classe{#1}\paren{#2,#3}}

\newcommand{\deriv}[1]{^{\paren{#1}}}

\usepackage{derivative}
\derivset{\pdv}[delims-eval=.)]
\derivset{\odv}[delims-eval=.)]

\DeclareMathOperator{\Arctan}{Arctan}
\DeclareMathOperator{\Arcsin}{Arcsin}
\DeclareMathOperator{\Arccos}{Arccos}
\DeclareMathOperator{\cotan}{cotan}
\DeclareMathOperator{\sh}{sh}
\DeclareMathOperator{\ch}{ch}
\DeclareMathOperator{\sg}{sg}
\DeclareMathOperator{\supp}{supp}
\DeclareMathOperator{\Supp}{Supp}
\DeclareMathOperator{\rg}{rg}
\DeclareMathOperator{\tr}{tr}

\newcommand{\Hom}[2]{\operatorname{Hom}\paren{#1,#2}}
\newcommand{\Pol}[2]{\operatorname{Pol}\paren{#1,#2}}
\newcommand{\Aut}[1]{\operatorname{Aut}\paren{#1}}
\DeclareDocumentCommand{\Vect}{O{} m}{\operatorname{Vect}_{#1}\paren{#2}}

\newcommand{\diag}[1]{\operatorname{diag}\paren{#1}}

\usepackage{abstract}
\addto\captionsfrench{\renewcommand{\abstractname}{\Large Introduction}}

\newcommand{\inv}{^{-1}}
\newcommand{\etoile}{^{*}}

\newcounter{orcounter}

\newenvironment{orlist}
{
\begin{array}{|l}
\setcounter{orcounter}{0}
}
{
\end{array}
}

\newcommand{\oritem}[1]{%
\ifthenelse{\theorcounter<1}{}{\\ \text{ou} \\}#1\stepcounter{orcounter}
}

\NewDocumentCommand{\orenv}{>{\SplitList{\\}}m}{%
\begin{orlist}\ProcessList{#1}{\oritem}\end{orlist}}

\newcounter{permuitemcounter}

\newcommand{\permuitem}[1]{%
\ifthenelse{\thepermuitemcounter<1}{}{&}#1\stepcounter{permuitemcounter}}

\NewDocumentCommand{\permu}{>{\SplitList{;}}m >{\SplitList{;}}m}{%
\begin{pmatrix}\setcounter{permuitemcounter}{0}\ProcessList{#1}{\permuitem} \\ \setcounter{permuitemcounter}{0}\ProcessList{#2}{\permuitem}\end{pmatrix}}

\NewDocumentCommand{\cycle}{>{\SplitList{;}}m}{%
\begin{pmatrix}\setcounter{permuitemcounter}{0}\ProcessList{#1}{\permuitem}\end{pmatrix}}

\usepackage{pgfplots}

\DeclareDocumentCommand{\pgcd}{o o}{
\IfNoValueTF{#1}{\operatorname{pgcd}}{\operatorname{pgcd}\paren{#1,#2}}
}

\DeclareDocumentCommand{\bezout}{o o}{
\IfNoValueTF{#1}{\operatorname{bezout}}{\operatorname{bezout}\paren{#1,#2}}
}

\usepackage{minted}
\newminted{python}{linenos, breaklines, breakanywhere, breakautoindent,tabsize=4,obeytabs}
\newenvironment{code}{\VerbatimEnvironment\begin{pythoncode}}{\end{pythoncode}}

\newcommand{\valp}[2]{v_{#1}\paren{#2}}

\newcommand{\fami}[1]{\mathscr{#1}}

\newcommand{\echange}{\leftrightarrow}

\newcommand{\trans}[1]{\prescript{t}{}{#1}}

\usepackage{mathdots}

\DeclareDocumentCommand{\detb}{O{\fami{B}}}{{\det}_{#1}}

\usepackage{cancel}

\usepackage{nicematrix}

\newcommand{\ps}[2]{\left\langle#1\tq#2\right\rangle}
\newcommand{\ortho}{^{\perp}}

\newcommand{\operp}{\mathrel{%
\begin{tikzpicture}[baseline=-0.25em]
\draw (0,0) circle (0.45em);
\draw (-0.38em,-0.25em) -- (0.38em,-0.25em);
\draw (0,-0.25em) -- (0,0.45em);
\end{tikzpicture}
}%
}

\usepackage{titletoc}
\dottedcontents{section}[5.5em]{}{3.2em}{1pc}

\newcommand{\bouleo}[2]{B\paren{#1,#2}}
\newcommand{\boulef}[2]{\conj{B}\paren{#1,#2}}
\newcommand{\sphere}[2]{S\paren{#1,#2}}

\newcommand{\vdv}[2]{\operatorname{D}_{#1}#2}

\newcommand{\egqd}[1]{\underset{#1}{=}}
\newcommand{\simqd}[1]{\underset{#1}{\sim}}

\newcommand{\tendqd}[1]{\xrightarrow[#1]{}}

\newcommand{\arr}[2]{A_{#2}^{#1}}
\newcommand{\comb}[2]{C_{#2}^{#1}}

\newcommand{\loiuniforme}[1]{\mathscr{U}\paren{#1}}
\newcommand{\loibernoulli}[1]{\mathscr{B}\paren{#1}}
\newcommand{\loibinomiale}[2]{\mathscr{B}\paren{#1,#2}}

\newcommand{\esp}[1]{\operatorname{E}\paren{#1}}
\newcommand{\vari}[1]{\operatorname{V}\paren{#1}}
\newcommand{\cov}[2]{\operatorname{Cov}\paren{#1,#2}}
\newcommand{\ecarttype}[1]{\sigma\paren{#1}}

\renewcommand{\O}[1]{\mathscr{O}\paren{#1}}
\renewcommand{\o}[1]{o\paren{#1}}

\setcounter{MaxMatrixCols}{200}

\newcommand{\Com}[1]{\operatorname{Com}#1}

\usepackage{microtype}

\newcommand{\sig}[1]{\epsilon\paren{#1}}

\newcommand{\legendeexercices}{\(\star\) Exercice proche du cours \\ \(\star\star\) Exercice de difficulté normale \\ \(\star\star\star\) Exercice difficile (voire très difficile) \\ \(+\) Exercice à faire en priorité \\}

\usepackage{accents}

\newcommand{\interieur}[1]{\accentset{\circ}{#1}}

\usepackage{etoolbox}

\DeclareFontFamily{U}{matha}{\hyphenchar\font45}
\DeclareFontShape{U}{matha}{m}{n}{
      <5> <6> <7> <8> <9> <10> gen * matha
      <10.95> matha10 <12> <14.4> <17.28> <20.74> <24.88> matha12
      }{}
\DeclareSymbolFont{matha}{U}{matha}{m}{n}
\DeclareFontSubstitution{U}{matha}{m}{n}

\DeclareFontFamily{U}{mathx}{\hyphenchar\font45}
\DeclareFontShape{U}{mathx}{m}{n}{
      <5> <6> <7> <8> <9> <10>
      <10.95> <12> <14.4> <17.28> <20.74> <24.88>
      mathx10
      }{}
\DeclareSymbolFont{mathx}{U}{mathx}{m}{n}
\DeclareFontSubstitution{U}{mathx}{m}{n}

\DeclareMathDelimiter{\vvvert}{0}{matha}{"7E}{mathx}{"17}
\DeclarePairedDelimiterX{\normesub}[1]
  {\vvvert}
  {\vvvert}
  {\ifblank{#1}{\:\cdot\:}{#1}}

\newcommand{\norme}[1]{\left\|\ifblank{#1}{\:\cdot\:}{#1}\right\|}

\newcommand{\tpt}{\text{pour tout }}
\newcommand{\Tpt}{\text{Pour tout }}

\newcommand{\Ann}[1]{\operatorname{Ann}\paren{#1}}

\newcommand{\difsym}{\mathbin{\Delta}}

\newcommand{\LRat}{\operatorname{LRat}}
\newcommand{\Regexp}{\operatorname{R}}
\newcommand{\loc}{\mathrm{loc}}

\begin{document}
\renewcommand{\labelitemi}{\(\triangleright\)}
\renewcommand{\labelenumi}{(\arabic{enumi})}

\everymath{\ds}

\maketitle

\begin{abstract}
Ce document réunit l'ensemble de mes cours d'Informatique de MPI, ainsi que les exercices les accompagnant. Le professeur était M. Carcenac. J'ai adapté certaines formulations me paraissant floues ou ne me plaisant pas mais le contenu pur des cours est strictement équivalent.

Les éléments des tables des matières initiale et présentes au début de chaque chapitre sont cliquables (amenant directement à la partie cliquée). C'est également le cas des références à des éléments antérieurs de la forme, par exemple, \guillemets{Démonstration 5.22}.
\end{abstract}

\dominitoc\tableofcontents

\part{Cours}

\chapter{Langages réguliers}

\minitoc

\note{À venir}


\chapter{Automates finis}

\minitoc

\note{À venir}


\chapter{Théorème de Kleene}

\minitoc

\note{À venir}


\part{Exercices}

\setcounter{chapter}{0}

\chapter{Langages réguliers}

\minitoc

\section*{Illustrations du cours}
\addcontentsline{toc}{section}{Illustrations du cours}

\begin{exo}[Exercice 1, vocabulaire sur les mots]
\begin{enumerate}
    \item Donner tous les préfixes propres du mot \(babbb\). \\
    \item Donner tous les facteurs du mot \(babbb\). \\
    \item Donner tous les sous-mots de \(babbb\).
\end{enumerate}
\end{exo}

\begin{corr}
\note{À venir}
\end{corr}

\begin{exo}[Exercice 2, opérations sur les langages]
On considère un alphabet \(\Sigma=\accol{a,b}\) et deux langages \(L_1=\accol{\epsilon,a,ba}\) et \(L_2=\accol{a,aa}\).

Donner les éléments des langages suivants : \(L_1\union L_2\), \(L_1^0\), \(L_1^1\), \(L_1L_2\) et \(L_2^3\).
\end{exo}

\begin{corr}
\note{À venir}
\end{corr}

\begin{exo}[Exercice 3, langages réguliers, expressions régulières]
On considère l'alphabet \(\Sigma=\accol{a,b}\).

Représenter par une expression régulière :

\begin{enumerate}[series=ex1.3]
    \item l'ensemble des mots composés de symboles \(a\) puis de symboles \(b\) ; \\
    \item l'ensemble des mots qui terminent par \(b\) ; \\
    \item l'ensemble des mots qui contiennent trois symboles \(a\) consécutifs.
\end{enumerate}

Exprimer en français l'ensemble des mots dénoté par l'expression régulière :

\begin{enumerate}[resume=ex1.3]
    \item \(r_4=a\paren{a\divise b}\etoile a\) ; \\
    \item \(r_5=\paren{a\divise b}\paren{a\divise b}\paren{a\divise b}\) ; \\
    \item \(r_6=a\etoile ba\etoile\).
\end{enumerate}
\end{exo}

\begin{corr}
\note{À venir}
\end{corr}

\section*{Compléments du cours}
\addcontentsline{toc}{section}{Compléments du cours}

\begin{exo}[Exercice 4, sous-mot]\thlabel{exo:definitionFormelleSousMot}
Formaliser la notion de sous-mot en utilisant :

\begin{enumerate}
    \item une suite de mots et de lettres ; \\
    \item une suite d'indices strictement croissante ; \\
    \item une fonction strictement croissante à valeurs entières.
\end{enumerate}
\end{exo}

\begin{corr}
\note{À venir}
\end{corr}

\begin{exo}[Exercice 5, propriétés sur les langages]\thlabel{exo:demonstrationsProprietesLangages}
Soient \(\Sigma\) un alphabet et \(L,L_1,L_2\) des langages sur \(\Sigma\).

Démontrer les propriétés suivantes :

\begin{enumerate}
    \item \(L.\ensvide=\ensvide.L=\ensvide\) \\
    \item \(L.\accol{\epsilon}=\accol{\epsilon}.L=L\) \\
    \item \(L^n.L^m=L^{n+m}\) \\
    \item \(L.\paren{L_1\union L_2}=L.L_1\union L.L_2\) \\
    \item \(L\etoile.L=L.L\etoile=L^+\) \\
    \item \(L\etoile=L^+\union\accol{\epsilon}\) \\
    \item \(\paren{L\etoile}\etoile=L\etoile\) \\
    \item \(L.\paren{L_1\inter L_2}\subset L.L_1\inter L.L_2\) et trouver un contre-exemple pour l'inclusion réciproque.
\end{enumerate}
\end{exo}

\begin{corr}
\note{À venir}
\end{corr}

\section*{Exercices}
\addcontentsline{toc}{section}{Exercices}

\begin{exo}[Exercice 6, quelques questions sur les mots]
Soit \(\Sigma\) un alphabet.

\begin{enumerate}
    \item Soient \(x,y,z\in\Sigma\etoile\). Montrer que \(xy=xz\ssi y=z\) et que \(yx=zx\ssi y=z\). \\
    \item Soient \(u,v,w\in\Sigma\etoile\) tels que \(u\leq_pw\) et \(v\leq_pw\). Montrer que \(u\leq_pv\) ou \(v\leq_pu\). \\
    \item Soient \(a,b\in\Sigma\) et \(u\in\Sigma\etoile\) tels que \(au=ub\). Montrer que \(a=b\) et \(u\in\accol{a}\etoile\). \\
    \item Soient \(x,y,u,v\in\Sigma\etoile\) tels que \(uv=xy\). Montrer qu'il existe un unique mot \(t\in\Sigma\etoile\) tel que \(u=xt\) et \(y=tv\) ou \(x=ut\) et \(v=ty\).
\end{enumerate}
\end{exo}

\begin{corr}
\note{À venir}
\end{corr}

\begin{exo}[Exercice 7, dénombrement]
Soit \(\Sigma\) un alphabet. On considère un mot \(m\in\Sigma\etoile\) de \(n\) symboles distincts.

\begin{enumerate}
    \item Combien le mot \(m\) a-t-il de préfixes, suffixes, facteurs et sous-mots ? \\
    \item Que peut-on dire si les symboles ne sont pas supposés distincts ?
\end{enumerate}
\end{exo}

\begin{corr}
\note{À venir}
\end{corr}

\begin{exo}[Exercice 8, préfixe : une relation d'ordre]
Soit \(\Sigma\) un alphabet.

\begin{enumerate}
    \item Justifier que \(\leq_p\) définit une relation d'ordre sur \(\Sigma\etoile\). \\
    \item Cet ordre est-il total ? \\
    \item Quel est l'ordre strict associé ?
\end{enumerate}
\end{exo}

\begin{corr}
\note{À venir}
\end{corr}

\begin{exo}[Exercice 9, racine carrée d'un langage]
Soit \(\Sigma\) un alphabet. Pour \(L\in\P{\Sigma\etoile}\), on définit \(\sqrt{L}=\accol{u\in\Sigma\etoile\tq u^2\in L}\).

\begin{enumerate}
    \item Que vaut \(\sqrt{L}\) pour \(L=\accol{\epsilon,a,b,aa,ab,bbb,bbbb}\) ? \\
    \item Montrer que \(\quantifs{\tpt L\in\P{\Sigma\etoile}}L\subset\sqrt{L^2}\). Trouver un contre-exemple montrant qu'il n'y a pas égalité.
\end{enumerate}
\end{exo}

\begin{corr}
\note{À venir}
\end{corr}

\begin{exo}[Exercice 10, quotient à gauche d'un langage]
Soit \(\Sigma\) un alphabet. Pour \(L\in\P{\Sigma\etoile}\) et \(a\in\Sigma\), on définit \(a\inv L=\accol{u\in\Sigma\etoile\tq au\in L}\).

\begin{enumerate}[series=ex1.10]
    \item Que vaut \(b\inv L\) pour \(L=\accol{\epsilon,a,b,ab,bab,bbbb}\) ?
\end{enumerate}

Pour \(R,L\in\P{\Sigma\etoile}\), on définit \(R\inv L=\accol{u\in\Sigma\etoile\tq\quantifs{\exists r\in R}ru\in L}\).

\begin{enumerate}[resume=ex1.10]
    \item Que vaut \(R\inv L\) pour \(R=\accol{a,bb}\) et \(L=\accol{\epsilon,a,b,aa,ab,bab,bbbb}\) ?
\end{enumerate}
\end{exo}

\begin{corr}
\note{À venir}
\end{corr}


\chapter{Automates finis}

\minitoc

\begin{exo}[Exercice 1, exemple]
On considère l'automate \(A_0\) ci-dessous :

\begin{center}
\begin{tikzpicture}
\node[state, initial above] (q1) {\(q_1\)};
\node[state, left of=q1] (q0) {\(q_0\)};
\node[state, below of=q0] (q3) {\(q_3\)};
\node[state, accepting, right of=q1] (q2) {\(q_2\)};
\node[state, accepting, below of=q2] (q4) {\(q_4\)};

\path[->] (q1) edge[above] node {\(a\)} (q0)
          (q1) edge[above] node {\(c\)} (q2)
          (q3) edge[left] node {\(b\)} (q0)
          (q3) edge[loop left] node {\(a\)} (q3)
          (q3) edge[bend right, below] node {\(c\)} (q1)
          (q2) edge[loop above] node {\(b\)} (q2)
          (q2) edge[bend left, right] node {\(a\)} (q4)
          (q2) edge[bend right, left] node {\(c\)} (q4);
\end{tikzpicture}
\end{center}

\begin{enumerate}
    \item Utiliser le vocabulaire approprié pour décrire l'automate \(A_0\). En particulier, donner sa définition formelle et représenter sa table de transition. \\
    \item Donner quelques exemples de mots acceptés par l'automate et de mots rejetés par l'automate. \\
    \item Donner son équivalent complet puis l'émonder.
\end{enumerate}
\end{exo}

\begin{corr}
\note{À venir}
\end{corr}

\begin{exo}[Exercice 2, déterminisation]
On considère l'automate \(A_1\) ci-dessous :

\begin{center}
\begin{tikzpicture}
\node[state, initial] (q0) {\(q_0\)};
\node[state, below right of=q0] (q2) {\(q_2\)};
\node[state, above right of=q2] (q1) {\(q_1\)};
\node[state, right of=q1, accepting] (q3) {\(q_3\)};

\path[->] (q0) edge[loop above] node {\(a\)} (q0)
          (q0) edge[above] node {\(a\)} (q1)
          (q0) edge[below] node {\(b\)} (q2)
          (q2) edge[loop below] node {\(a,b\)} (q2)
          (q2) edge[above] node {\(a\)} (q1)
          (q1) edge[above] node {\(b\)} (q3);
\end{tikzpicture}
\end{center}

\begin{enumerate}
    \item Justifier que l'automate \(A_1\) n'est pas déterministe. \\
    \item Donner un mot \(m\) pouvant donner lieu à au moins deux calculs, l'un aboutissant à un état final, l'autre aboutissant à un état non-final. Ce mot est-il reconnu par l'automate ? \\
    \item Déterminer cet automate pour obtenir \(A_{1\,\det}\). Le mot \(m\) est-il reconnu par \(A_{1\,\det}\) ?
\end{enumerate}
\end{exo}

\begin{corr}
\note{À venir}
\end{corr}

\begin{exo}[Exercice 3, automates \guillemets{évidents}]
On considère l'alphabet \(\Sigma=\accol{a,b,c}\).

Pour chaque ensemble de mots \(L_i\), donner une expression régulière \(r_i\) dénotant \(L_i\) et dessiner un automate \(A_i\) (déterministe ou non) reconnaissant \(L_i\) :

\begin{enumerate}
    \item \(L_1\) : les mots de trois lettres qui commencent par \(a\) ; \\
    \item \(L_2\) : les mots qui ne contiennent qu'un seul \(b\) ; \\
    \item \(L_3\) : les mots qui ont \(ab\) pour préfixe ; \\
    \item \(L_4\) : les mots qui ont \(bac\) pour suffixe ; \\
    \item \(L_5\) : les mots qui ont \(abba\) pour facteur ; \\
    \item \(L_6\) : les mots qui ont \(abba\) pour sous-mot.
\end{enumerate}
\end{exo}

\begin{corr}
\note{À venir}
\end{corr}

\begin{exo}[Exercice 4, donner un sens aux états]
On considère l'alphabet \(\Sigma=\accol{0,1}\). Représenter un automate reconnaissant :

\begin{enumerate}
    \item l'ensemble des mots contenant un nombre pair de symboles ; \\
    \item l'ensemble des mots tels que le nombre d'occurrences de \(1\) soit multiple de \(3\) ; \\
    \item l'ensemble des mots \(m\) tels que \(\abs{m}_0-\abs{m}_1\equiv1\croch{9}\).
\end{enumerate}
\end{exo}

\begin{corr}
\note{À venir}
\end{corr}

\begin{exo}[Exercice 5, automate déterministe complet]
On définit un automate déterministe incomplet comme un quintuplet \[A=\paren{Q,\Sigma,q_i,Q_F,\delta:D\subset Q\times\Sigma\to Q}.\]

\begin{enumerate}
    \item Définir formellement l'automate complet \(A_c\) associé. \\
    \item Justifier formellement que ces deux automates sont équivalents.
\end{enumerate}
\end{exo}

\begin{corr}
\note{À venir}
\end{corr}

\begin{exo}[Exercice 6, automate produit]
On considère l'alphabet \(\Sigma=\accol{a,b}\).

\begin{enumerate}
    \item Représenter un automate déterministe \(A\) reconnaissant les mots ayant un nombre pair de symboles. \\
    \item Représenter un automate déterministe \(A\prim\) reconnaissant les mots composés de symboles \(a\) puis de symboles \(b\). \\
    \item En utilisant l'automate produit, représenter un automate reconnaissant les mots contenant un nombre pair de symboles qui sont composés de symboles \(a\) puis de symboles \(b\). Donner un sens à chaque état de l'automate obtenu.
\end{enumerate}
\end{exo}

\begin{corr}
\note{À venir}
\end{corr}

\begin{exo}[Exercice 7, déterminisation \guillemets{explosive}]
On considère l'alphabet \(\Sigma=\accol{a,b}\) et \(n\in\Ns\). On pose \(L_n\) le langage des mots ayant un \(a\) \guillemets{\(n\) lettres avant la fin du mot}. Par exemple, \(L_1\) est le langage des mots ayant un \(a\) une lettre avant la fin, \ie en dernière lettre ; \(L_2\) est le langage des mots ayant un \(a\) deux lettres avant la fin, \ie en avant-dernière lettre.

\begin{enumerate}[series=ex2.7]
    \item Donner une expression régulière \(e_n\) dénotant \(L_n\). \\
    \item Donner un automate \(A_n\) non-déterministe à \(n+1\) états qui reconnaît \(L_n\). On nommera \(q_0,\dots,q_n\) les états. \\
    \item Déterminiser \(A_3\). Combien d'états l'automate obtenu a-t-il ? \\
    \item Trouver un mot de trois lettres permettant, depuis \(q_0\), d'atteindre exactement l'ensemble des états \(\accol{q_0,q_2}\).
\end{enumerate}

Plus généralement, on va montrer que l'automate déterminisé associé à \(A_n\) possède \(2^n\) états.

Pour \(m\in\Sigma\etoile\) et \(q\) un état, on note \(A\paren{m}\) l'ensemble des états de \(A_n\) accessibles à la lecture du mot \(m\) et \(L_{\to q}\) le langage reconnu par l'automate \(A_n\) légèrement modifié de sorte que \(q\) soit son unique état final.

\begin{enumerate}[resume=ex2.7]
    \item Décrire en français les langages \(L_{\to q_0},L_{\to q_1},\dots,L_{\to q_n}\). \\
    \item Montrer que pour \(m\in\Sigma\etoile\), \(A\paren{m}=\accol{q\tq m\in L_{\to q}}\). \\
    \item Montrer que pour chaque ensemble d'états \(Q\) contenant \(q_0\), il existe un mot \(m\) tel que \(\abs{m}=n\) et \(A\paren{m}=Q\). \\
    \item En déduire, le nombre d'états de l'automate déterminisé associé à \(A_n\).
\end{enumerate}

Peut-être que la méthode de déterminisation construit un automate inutilement compliqué... Est-il possible de trouver un automate déterministe de moins de \(2^n\) états qui reconnaisse \(L_n\) ? Ce qui suit montre que non.

Supposons qu'il existe un automate déterministe complet ayant strictement moins de \(2^n\) états qui reconnaît \(L_n\).

\begin{enumerate}[resume=ex2.7]
    \item Justifier qu'il existe deux mots \(u\) et \(v\) de longueur \(n\) dont la lecture aboutit au même état. \\
    \item En considérant la première lettre qui différencie ces deux mots, et en les rallongeant, aboutir à une contradiction.
\end{enumerate}
\end{exo}

\begin{corr}
\note{À venir}
\end{corr}

\begin{exo}[Exercice 8, lemme d'Arden et application]
Lemme d'Arden : si \(L\) est un langage vérifiant l'égalité \(L=E.L\union F\) où \(E,F\) sont des langages tels que \(\epsilon\not\in E\), alors \(L=E\etoile.F\).

On peut rencontrer la formulation suivante : \(E\etoile.F\) est l'unique solution de l'équation \(X=E.X\union F\) d'inconnue \(X\), lorsque \(\epsilon\not\in E\).

\begin{enumerate}[series=ex2.8]
    \item Démontrer le lemme d'Arden (version équation). Montrer que le langage proposé est bien solution de l'équation puis montrer que c'est l'unique solution de l'équation.
\end{enumerate}

Application : le lemme d'Arden peut servir à exprimer le langage reconnu par un automate sous forme d'expression régulière.

Exemple : on considère l'automate \(A\) ci-dessous :

\begin{center}
\begin{tikzpicture}
\node[state, initial] (q1) {\(q_1\)};
\node[state, below right of=q1, accepting] (q3) {\(q_3\)};
\node[state, above right of=q3] (q2) {\(q_2\)};

\path[->] (q1) edge[above] node {\(a\)} (q2)
          (q1) edge[below] node {\(b\)} (q3)
          (q3) edge[loop below] node {\(a,b\)} (q3)
          (q2) edge[loop right] node {\(b\)} (q2)
          (q2) edge[below] node {\(a\)} (q3);
\end{tikzpicture}
\end{center}

Pour \(i\in\accol{1,2,3}\), on note \(L_i\) le langage reconnu en prenant \(q_i\) pour état initial.

\begin{enumerate}[resume=ex2.8]
    \item Pour chaque état \(q_i\), établir une relation entre les \(L_i\). \\
    \item À l'aide du lemme d'Arden, donner une expression pour les langages \(L_i\). \\
    \item En déduire le langage reconnu par l'automate.
\end{enumerate}
\end{exo}

\begin{corr}
\note{À venir}
\end{corr}

\begin{exo}[Exercice 9, langages non-rationnels]
Montrer que les langages ci-dessous ne sont pas rationnels :

\begin{enumerate}
    \item \(L_1=\accol{a^nb^n\tq n\in\N}\) ; \\
    \item \(L_2=\accol{a^p\tq p\text{ est premier}}\) ; \\
    \item \(L_3=\accol{m\tq\abs{m}_a=\abs{m}_b}\).
\end{enumerate}
\end{exo}

\begin{corr}
\note{À venir}
\end{corr}

\begin{exo}[Exercice 10, vers l'automate de Glushkov : automate local]
Montrer que la procédure de construction de l'automate local \(A_\loc\) associé à un langage local \(LL\) est correcte, \cad que \(\Lendo{A_\loc}=LL\).
\end{exo}

\begin{corr}
\note{À venir}
\end{corr}

\begin{exo}[Exercice 11, vers l'automate de Glushkov : ensembles \guillemets{caractéristiques}]
Pour les langages suivants, déterminer les ensembles caractéristiques \(P\paren{L}\), \(S\paren{L}\), \(F\paren{L}\) et \(N\paren{L}\), puis déterminer si le langage est local :

\begin{enumerate}
    \item \(L_1=\Lendo{abab}\) ; \\
    \item \(L_2=\Lendo{abc}\) ; \\
    \item \(L_3=\Lendo{a\etoile}\) ; \\
    \item \(L_4=\Lendo{\paren{ab}\etoile}\) ; \\
    \item \(L_5=\Lendo{\paren{ab}\etoile a\etoile}\).
\end{enumerate}
\end{exo}

\begin{corr}
\note{À venir}
\end{corr}

\begin{exo}[Exercice 12, vers l'automate de Glushkov : propriétés des langages linéaires]
\begin{enumerate}
    \item Soit \(L\) un langage local. Exprimer \(P\paren{L\etoile}\), \(S\paren{L\etoile}\) et \(F\paren{L\etoile}\) en fonction de \(P\paren{L}\), \(S\paren{L}\) et \(F\paren{L}\) ; montrer que \(L\etoile\) est un langage local. \\
    \item De même, montrer que l'union de deux langages locaux sur des alphabets distincts est un langage local. \\
    \item De même, montrer que la concaténation de deux langages locaux sur des alphabets distincts est un langage local. \\
    \item Montrer que \(\ensvide\), \(\accol{\epsilon}\) et \(\accol{a}\) avec \(a\in\Sigma\) sont des langages locaux. \\
    \item En déduire que le langage dénoté par une expression régulière linéaire est local.
\end{enumerate}
\end{exo}

\begin{corr}
\note{À venir}
\end{corr}


\chapter{Théorème de Kleene}

\minitoc

\note{À venir}

\end{document}
